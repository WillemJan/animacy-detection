\documentclass[a4paper,UKenglish]{oasics}

\usepackage{microtype, graphicx, booktabs, multirow, color, colortbl}

\usepackage{gb4e, rotating}
\bibliographystyle{plain}

\definecolor{Gray}{gray}{0.9}
\usepackage[utf8]{inputenc}

\title{Animacy Detection in Stories}
\author[1]{Folgert Karsdorp}
\author[2]{Marten van der Meulen}
\author[3]{Theo Meder}
\author[4]{Antal van den Bosch}
\affil[1]{Meertens Institute\\
  Amsterdam, The Netherlands\\
  \texttt{fbkarsdorp@fastmail.nl}}
\affil[2]{Meertens Institute\\
  Amsterdam, The Netherlands\\
  \texttt{marten.van.der.meulen@meertens.knaw.nl}}
\affil[3]{Meertens Institute\\
  Amsterdam, The Netherlands\\
  \texttt{theo.meder@meertens.knaw.nl}}
\affil[4]{Radboud University\\
  Nijmegen, The Netherlands\\
  \texttt{a.vandenbosch@let.ru.nl}}
%\authorrunning{F. Karsdorp and M. van der Meulen and A. van den Bosch}
%\Copyright{Folgert Karsdorp and Marten van der Meulen and Antal van
%  den Bosch}
\authorrunning{F. Karsdorp and M. van der Meulen and T. Meder and A. van den Bosch}
\Copyright{Folgert Karsdorp and Marten van der Meulen and Theo Meder
  and Antal van den Bosch}
\subjclass{I.2.7 Natural Language Processing}
\keywords{animacy detection, deep learning, folklore}

%Editor-only macros:: begin (do not touch as author)%%%%%%%%%%%%%%%%%%%%%%%%%%%%%%%%%%
\serieslogo{}%please provide filename (without suffix)
\volumeinfo%(easychair interface)
  {Billy Editor and Bill Editors}% editors
  {2}% number of editors: 1, 2, ....
  {Conference/workshop/symposium title on which this volume is based on}% event
  {1}% volume
  {1}% issue
  {1}% starting page number
\EventShortName{}
\DOI{10.4230/OASIcs.xxx.yyy.p}% to be completed by the volume editor
% Editor-only macros::end %%%%%%%%%%%%%%%%%%%%%%%%%%%%%%%%%%%%%%%%%%%%%%%


\begin{document}
\maketitle

\begin{abstract}
  Existing animacy detection systems, while scoring well, seem to be
  out of sync with current (psycho)linguistic research, which takes a
  more dynamic approach to animacy. This paper presents a
  linguistically uninformed computational model for animacy
  classification. This model makes use of word $n$-grams in
  combinations with lower dimensional word embedding representations
  that were learned from a web-scale corpus. We compare the model to a
  number of linguistically informed models that use features such as
  dependency tags and show competitive results.
\end{abstract}
%aanpassen op basis van embedding verhalenbank-evaluatie%
\section{Introduction}

For almost all species in the world, the capacity to distinguish
animate objects from inanimate objects is essential to their
survival. Those objects could be prey, for example, or predators or
mates. The fundamental nature that the distinction between animate and
inanimate has for humans is reflected in the fact that this division
is acquired very early in life \cite{opfer:02}: children of less than
6 months old are well able to distinguish the two categories from one
another. Moreover, recent brain research shows that the distinction
appears in the organization of the brain (e.g.~\cite{gao:12}). For
some researchers, this provides evidence for the idea that the
division between animate and inanimate is an innate part of how we see
the world.

Although animacy may be a scalar rather than a
strictly categorical distinction (see e.g. the animacy hierarchy in
\cite{comrie:89} and research such as \cite{rosenbach:08}), the
animate/inanimate distinction is traditonally taken as fixed with regard to lexical
items: something is either animate (e.g. a human) or not (e.g. a chair). This standpoint has been challenged, however, by researchers
from different fields. Firstly, it has long been established in
linguistic typology that not all languages award animacy to the same
entities in different grammatical categories: as \cite{comrie:89} notes, for
example, English, and many other languages, distinguishes
between human and not-human in the choice of pronouns, whereas Russian
distinguishes between animate (entailing humans and animals) versus
non-animate (entailing everything else) in its interrogative pronouns. This indicates different
subdivisions of animacy in the respective languages. Secondly,
philosophers such as Daniel Dennett support the view that animacy and
aliveness are to be treated as epistemological stances rather than
fixed states in the world: not ineffable qualia but behavioral
capacity defines our stance towards objects \cite{dennett:96}. In
other words, depending on whether people \textit{think} that an object
is animate, they utilize different cognitive strategies to explain and
predict the actions of those objects. Finally, evidence from
psycholinguistic research has accumulated to support this view of
animacy as a cognitive viewpoint rather than an extra-perceptive
absolute. \cite{nieuwland:05} for example show that test subjects
readily accept animate behavior from inanimate objects within the
proper contexts, and \cite{vogels:13} moreover emphasize the relation
between animacy and motion, showing that factors such as
self-propelment play a crucial role in recognizing and/or awarding
animacy to certain objects. This is exemplified in the opening of this
well-known storyhttp://www.verhalenbank.nl/items/show/9636:

\begin{quotation} {\it A farmer bought a pancake on the market. Once
    he got home, the farmer was hungry and began to bake the
    pancake. The farmer tried one of his skillful flipping techniques,
    but he failed and the pancake fell on the ground. Coincidentally,
    the door of the kitchen was open and the pancake rolled out to the
    field, as hard as he could\ldots}
\end{quotation}

Although initially, based on their knowlegde of the world, readers
will regard the pancake as inamimate, the self-propelled motion verb
`rolled' initiates our shift towards an animate interpretation of the
pancake. As readers (or listeners) of a story, we choose to view
participating objects at varying levels of abstraction in order to
predict their behavior. Dennett \cite{dennett:96} defines three levels
of abstraction: (1) the \textit{physical stance}, (2) the
\textit{design stance} and (3) the \textit{intentional stance}. The
physical stance deals with predictions about objects given their
physical properties. The design stance is slightly more abstract and
deals with concepts like purpose, function or design. Finally the
intentional stance is concerned with belief, thinking and
intentions. These are all cognitive strategies we use to predict and
explain the actions of objects in our environment. Interestingly, in
the process of reading the opening of the story about the fleeing
pancake, readers and listeners experience the transition from one
strategy to the next quite clearly. Initially, the pancake is
interpreted from a physical stance, or perhaps the more abstract
design stance in terms of the purpose (i.e. to stave off hunger). It
is only at the last adverbial phrase `as hard as he could' that we
start to wonder whether we should adopt to the yet more abstract
intentional stance and consider the pancake to be a rational agent.

Given the fundamental nature of the distinction between animate and
inanimate, it is perhaps not too surprising that it has proven to be
useful in a variety of natural language processing tasks dealing with
e.g.\ anaphora resolution and dependency parsing
\cite{orasan:07,lee:13,ovr:niv:07}. Existing methods for the automatic
labeling of text for animacy are usually either rule based or apply
machine learning techniques, or make a combination of both. Common to
most approaches is the fact that they make use of semantic lexicons
with information about animacy, as well as syntactic cues in a
text. Both feature types are relatively costly to obtain as they
require large vocabularies or syntactic parsing systems, which, with
the exception of a few languages, are not readily available.

In this paper we present a new linguistically uninformed model to
automatically label texts for animacy. We show that we can (mostly) do
away with features that require syntactic parsing or semantic lexicons
while still yielding competitive performance. We focus on labeling
animacy in stories because stories pose some interesting problems to
automatic systems of animacy recognition. As the example of the
fleeing pancake already made clear, in stories quite some entities can exhibit all behavioral
capacities to consider them animate and rational agents, even though they are inanimate in the `real' world. Another
example is the \textit{Sorcerer's Apprentice} sequence in Walt
Disney's famous \textit{Fantasia}, in which brooms display the ability
to collect buckets of water. Such examples, in which otherwise as
inanimatedly labeled entities such as pancakes and brooms act as animate beings, make a
clear case for developing dynamic, data driven systems that do not
rely too much on static and fixed world knowledge, but rather on
immediate context.

The rest of this paper is structured as follows. We will start with a
short overview of existing techniques for automatically labeling
animacy in texts, including the definitions of animacy used in these
papers (\S\ref{sec:previous-work}). After a description of the corpus
used in our study and how the annotations of the corpus have been
established (\S\ref{sec:data}), we will give an account of our
computational models in Section \ref{sec:models}. We report on the
empirical results in Section \ref{sec:results}. Next, we provide an evaluation on a larger dataset, while also showing a real world application of our animacy detection system \ref{sec:listing}.
The last section offers our conclusions and possible directions for future research.


\section{Previous Work}\label{sec:previous-work}

A handful of papers deal with automatic animacy detection. Most
approaches make use of rule-based systems or machine learning systems
with morphological and syntactic features. \cite{evans:00} present a
rule-based system that makes use of the lexical-semantic database
WordNet. They label each synset in WordNet for animacy. Using a
variety of rules to detect the head of an NP, they use the fraction of
synsets in which a particular noun occurs to arrive at a
classification for animacy. \cite{orasan:01} extend their previous
algorithm by first determining the animacy of senses from WordNet on
the basis of an annotated corpus. They then apply a $k$-nearest
neighbor classifier using a number of lexical and syntactic features
and features derived from WordNet to arrive at a final animacy
classification.

\cite{ovrelid:04,ovrelid:05,ovrelid:06,ovrelid:08,ovrelid:09} present
a number of animacy classifiers that make use of syntactic and
morphological features. These features include the frequency of
analysis of the noun as `subject' or `object', the frequency of the
occurence of a noun in a passive $by$-phrase, and the frequency of the
noun as a subject followed by either animate personal pronouns or
inanimate personal pronouns. These features are then aggregated for
each lemma after which a machine learning system (decision tree or
$k$-nearest neighbor classifier) is trained. In \cite{bowman:12} a
similar approach is presented. In this study a Maximum Entropy
classifier is trained on the basis of three feature types: (1)
bag-of-words with and without their corresponding Part-of-Speech tags,
(2) internal syntactic features such as the syntactic head and (3)
external syntactic features that describe the dependency relation of a
noun to a verb (i.e.\ subject relation, object relation etc.)  This is
the only study that makes use of a fully labeled corpus for animacy.
Partially related to animacy detection, \cite{karsdorp:12}
attempt to extract the cast (i.e.\ all characters) from a
story. Similar to \cite{bowman:12} they rely on dependency tags to
extract the subjects of direct and indirect speech.

\cite{bloem:13} present a model that attempts to generalize the
animacy information in a lexical-semantic database of Dutch by
augmenting `non-ambiguous' animate entries with contextual information
from a large treebank of Dutch. They apply a $k$-nearest neighbor
algorithm with distributional lexical features that aim to capture the
association using measures like mutual information between a verb or
adjective and a particular noun. The idea is that nouns that occur in
similar contexts as animate nouns are more likely to be animate than
nouns that occur more frequently in contexts similar to inanimate nouns.

\cite{moore:13} present an approach that combines a number of animacy
classifiers in a voting scheme and aims at an interpretable and
correctable model of animacy classification. They combine a variety of
classifiers such as the WordNet-based approach of \cite{evans:00},
dictionary sources and systems for named entity recognition.

The approaches mentioned above present us with a number of
problems. Firstly, almost all of them heavily rely on costly,
linguistically informed features derived from lexical-semantic
databases or syntactic parsing. For most languages in the world,
however, we cannot rely on these resources, either because they do not
exist, or because their performance is insufficient.  Second, animacy
detection is often seen as a useful feature for a range of natural
language processing techniques, such as anaphora resolution and
syntactic parsing. The mutual dependence between these techniques and
animacy detection, however, leads to a chicken and egg
situation.

Another major problem with the approaches above is, as was said above,
that they are lemma-based, which means that the models are generally
insensitive to different usages of a particular word in particular
contexts. In other words, in most of the literature on automatic
animacy detection, a static, binary distinction is made between
animate and inanimate. \cite{bowman:12} for example, define objects as
animate if they are alive and have the ability to move under their own
will. \cite{orasan:07} define animacy in the context of anaphora
resolution: something is animate ``if its referent can also be
referred to using one of the pronouns he, she, him, her, his, hers,
himself, herself, or a combination of such pronouns
(e.g. his/her)''. However, as was explained above, these definitions
are not necessarily in line with current linguistic and neurological
research, and seemingly ignore findings from research such as
\cite{nieuwland:05}. Similarly, they are not particularly applicable
to the rich and wondrous entities that live in the realm of
stories. As was shown above, although a pancake is typically not an
animate entity, its animacy depends on the story in which it appears,
and even within the story the animacy may change. To accomodate this
possibility, we therefore choose to define animacy in terms of
Dennett's intentional stance, which is more dynamic, and which
ultimately comes down to the question whether ``you decide to treat
the object whose behavior is to be predicted as a rational agent''
\cite[pp. 17]{dennett:96}. Our system for animacy detection therefore
needs to be dynamic, data driven, and token based. It cannot rely too
heavily on static and fixed world knowledge.

\section{The Data, Annotation and Preprocessing}\label{sec:data}

To develop this dynamic, data driven system we use a corpus of Dutch
folktales. One of the reasons to use folk tales is that, as
\cite{vogels:13} note, `In cartoons or fairy tales [\ldots] inanimate
entities or animals are often anthropomorphized', which means that the
material could yield interesting cases of unexpected animacy, as is
the case with the pancake in the abovementioned story. 

Our initial corpus consists of 74 Dutch stories from the collection
\textit{Volkssprookjes uit Nederland en Vlaanderen}, compiled by
\cite{sinninghe:78}. The collection is made up of Dutch retellings of
popular and widespread stories, including such tales as \textit{The
  Bremen Town Musicians} (ATU 130) and \textit{The Table, the Ass, and
  the Stick } (ATU 563), as well as lesser-known stories such as
\textit{The Singing Bone} (ATU 780) and \textit{Cock, Hen, Duck, Pin,
  and Needle on a Journey} (ATU 210). This last story is again a clear
example where otherwise inanimate objects are animated in a folk
story, as it concerns the adventures of several household items, such
as a \textit{pin}, a \textit{hackle}, an \textit{egg}, and a
\textit{whetstone}. A digital version of the collection is available
in the Dutch Folktale Database from the Meertens Institute (corpus
SINVSUNV.20E).\footnote{See \url{http://www.verhalenbank.nl}} Using a
single collection for our corpus presents us with a helpful
homogeneity with regard to the editor, length of the stories (which is
between X and Y words) and language use. On the other hand, the
collection displays diversity as well in the choice of the stories,
which entail fairy tales, legends, and sagas.

All together, the corpus consists of 74504 words, from 5549 unique
words. Using the annotation tool brat (brat rapid annotation tool), an
online environment for collaborative editing\footnote{Available here:
  \url{http://brat.nlplab.org}}, two annotators labeled words for
animacy, within the context of the story.\footnote{On the basis of
  five stories that were annotated by both annotators we computed an
  inter-annotator agreement score (Cohen's Kappa) of $K=0.95$.} All
unlabeled words were implicitly considered to be inanimate. The
following sentence provides an example annotation.
\begin{exe}
\ex
\gll Jij smid, jij bent de sterkste; hou je vast aan de bovenste
takken, en dan ga jij kleermaker aan zijn benen hangen en zo gaan we maar door\\
    \textsc{animate} \textsc{animate} \textsc{animate} {} {} {} {} \textsc{animate} {} {} {} {} {} {} {} {} \textsc{animate} \textsc{animate}
    {} \textsc{animate} {} {} {} {} {} \textsc{animate} {} {} \\
\trans `You, blacksmith, you are the strongest; hold on to the upper
branches and then, you, tailor, will grab his legs and so we go on\ldots'
\end{exe}

Because we interpreted animacy within the context of the story, the
same lexical item could be labeled differently in different stories.
For example, in the abovementioned example of the pancake, which
occurs in SINVS076 in our corpus, the pancake is tagged consistently
as `animate'. In another story, SINVS042, where at one point a soldier
is baking pancakes, the pancakes do not act, and are thus not labeled
as `animate'. The following sentences show how this was employed in
practice.
\begin{exe}
\ex
\gll Terwijl hij de pannekoek bakte, keek hij naar het ding, dat uit de schouw gevallen was \\
    {} \textsc{animate} {} {} {} {} \textsc{animate} {} {} {} {} {} {} {} {} {} {}\\
\trans `While he was baking the pancake, he looked at the thing, which had fallen from the hearth\ldots'
\end{exe}
\begin{exe}
\ex
\gll Toevallig stond de deur van de keuken open en de pannekoek rolde naar buiten, het veld in, zo hard hij maar kon.\\
    {} {} {} {} {} {} {} {} {} {} \textsc{animate} {} {} {} {} {} {} {} {} {}\textsc{animate} {} {} \\
\trans `Coincidentally the door of the kitchen was open and the pancake rolled outside, into the field, as fast as it could'
\end{exe}

This annotation resulted in 11542 animate tokens of 743 types, while
implicitly yielding 62926 inanimate tokens from 5011 unique inanimate
words. Because of our context-dependent approach, some words, such as
\textit{pancake} and \textit{egg}, occurred in both animate types as
inanimate types, because they were labeled as both animate and
inanimate in some cases in our corpus. It is telling that of the
animate tokens 4627 (40,1\%) were nouns, while only 6878 of the
inanimate tokens (11 \%) are nouns. This shows that being a noun is
already somewhat of an indication for animacy.
%hoeveel woorden hebben we eigenlijk die dubbel voorkomen, zoals pannekoek en ei? Leuk om te melden%
After tokenization with the tokenization module of the Python software
package Pattern \cite{smedt:12} we fed all stories to the
state of the art syntactic parser for Dutch, Alpino \cite{bouma:01}.
%dit kan iets uitgebreider eventueel%

\section{Experimental Setup}\label{sec:models}
This section describes our experimental setup including the features
used, the machine learning models we applied, and our methods of
evaluation.\footnote{The data set and the code to perform the
  experiments are made available online at
  \url{https://fbkarsdorp.github.io/animacy-detection}}.

\subsection{Task description}

We formulate the problem of animacy detection as a classification
problem where the goal is to assign a label at word level, rather than
at lemma level. This label indicates whether the word is classified as
animate or inanimate.

\subsection{Evaluation}

Animate words are far outnumbered by inanimate words in our collection
(see \S\ref{sec:data}). Reporting accuracy scores would therefore
provide skewed results, strongly favoring the majority category. The
relative rarity of animate words makes evaluation measures such as the
well-known $F1$-score more appropriate. For this reason, we report
on the precision, recall and $F1$-score~\cite{rijsbergen:79} for all experiments. As a
second measure of evaluation, we compute the Area Under the Precision
Recall Curve (AUC_{PR}) which also takes into account possible class
imbalance. Also, while in most of the literature on animacy detection
results are only presented for the classification of nouns or noun
phrases, we will, while reporting on nouns and noun phrases as well, additionally report on the
results for all words in a text.

In real world applications, an animacy detection system will most
likely be faced with completely new texts instead of single words. It
is therefore important to construct a training and test procedure in
such a way that it mimics this situation as closely as possible. If we
would, for example, make a random split of 80\% of the data for
training and 20\% for testing on the word level, we run the risk of
mixing training data with test data, thereby making it too easy for a
system to rely on words it has seen from the same
text. \cite{bowman:12} fall into this trap by making a random split in
their data on the sentence level. In such a setup, it is highly likely
that sentences from the same document are present in both the training
data and the test data, making their evaluation unrealistic. To
circumvent this problem, we split the data at the story level. We make
use of 10-fold cross-validation, where we randomly select 10\% of the
stories for testing and train on the remaining 90\%.

\subsection{Features}

We explore a range of different features and feature combinations
including lexical features, morphological features, syntactic features
and semantic features.

\subsubsection{Lexical features}
We take a sliding window approach where for each focus word (i.e. the
word for which we want to predict whether it is animate or not) we
extract both $n$ words to the left and $n$ words to the right, as well
as the focus word itself. In all experiments we set $n$ to 3. In
addition to the word forms, for each word in a window we also extract
its lemma.

\subsubsection{Morphological Features}
For each word we extract its part-of-speech tag. For reasons of
comparability we choose to use the tags as provided by the output of
the syntactic parser Alpino, instead of a more specialized
part-of-speech tagger. Again, we take a sliding window approach and
extract the part-of-speech tags for three words left and right of the
focus word, as well as the tag of the focus word itself.
%waarom Alpino voor comparibility, omdat we dat ook gebruikten in het taggen van het corpus?%
\subsubsection{Syntactic Features}
We extract the dependency tag for each word and its $n=3$ neighbors to
the right and to the left as provided by the syntactic parser
Alpino. Animate entities tend to take the position of subject or
object in a sentence which is why this feature is expected and has
proven to perform rather well.

\subsubsection{Semantic Features} 
The most innovative feature we have
included in our model is concerned with semantic similarity. In his
\textit{Philophische Untersuchungen} Wittgenstein already suggests
that ``Die Bedeutung eines Wortes ist sein Gebrauch in der Sprache''
(PI 43). This is reflected by the well-known insight in computational linguistics that the
meaning of words can be approximated by comparing the linguistic
contexts in which words appear, is an implementation of this very
idea. The idea is very simple, yet powerful: words that often
co-appear with the same set of words, will have a more similar
meaning. Recently, there has been a lot of interest in procedures that
can automatically induce so-called `word embeddings' from large,
unannotated collections of texts
(e.g.~\cite{mikolov:13,pennington:14}). These models typically attempt
to learn a lower dimensional vector for each word in the vocabulary
which captures the typical co-occurrence patterns of the word in the
corpus. The similarity between words can then be approximated by
applying similarity metrics, such as the cosine metric, to these
vectors of word embeddings.

We have trained word embeddings with 300 dimensions using the popular
skip-gram architecture \cite{mikolov:13} on the Dutch corpus of COW
(COrpora from the Web).\footnote{The word embedding vectors have been
  published online at XXX.} COW is a collection of linguistically
processed web corpora for English, Dutch, Spanish, French, Swedish and
German \cite{schaefer:12}. The 2014 Dutch corpus contains 6.8 billion
word tokens. The idea behind using the word embeddings is that
similarities between animate words can be estimated by inspecting the
context in which they occur. From this follows, for example, that the word
embeddings of an animate word are more similar to those of other
animate words, as opposed to the embeddings of inanimate words.

\begin{figure}[t]
\centering
\includegraphics[width=\textwidth]{pca.pdf}
\caption{Two-dimensional PCA projection of the 300 dimensional word embedding
vectors for a number of animate and inanimate words. The horizontal line
illustrates the separabilty between the two classes in the first dimension.}
\label{fig:pca}
\end{figure}

To give an illustration of this idea, in Figure \ref{fig:pca} we depict a
two-dimensional Principle Component Analysis (PCA) projection of the 300 dimensional
word embedding vectors for a number of typically animate and typically inanimate
 words. The horizontal grey line in the plot illustrates the separability of
 the animate and inanimate words in the first dimension of the PCA projection.
 It is interesting to observe that the less animate entity \emph{ghost} is the
 one closest to all other inanimate entities. Likewise, words such as \emph{castle},
 \emph{house} or \emph{car} are often used in figurative language (metonomy),
 for example to refer to the people owning or living in the castle. Perhaps this
 ambiguous animacy position is reponsible for their position in the first dimension
 close to real animate entities.
%dit stukje hierboven kan nog iets worden uitgewerkt, eventueel met voorbeelden%
\subsection{Models}
We employ a Maximum Entropy classifier with L2 regularization as
implemented in \cite{sklearn}. In all experiments, we set the
regularization strength parameter $C$ to 1.

We compare nine models in which we make use of different feature
combinations: (1) words, (2) words and Part-of-Speech tags, (3)
words, Part-of-Speech tags and lemmata, (4) words, Part-of-Speech
tags, lemmata and dependency tags, (5) word embeddings and (6-9) the
features in model 1 to 4 plus word embeddings.

Although our background corpus is sufficiently large to cover most
words in an unseen text, there will always be rare words for which we
do not have learned word embeddings. Therefore, in order to
effectively make use of the word embedding vectors, we need a way to
deal with out-of-vocabulary items. We adopt a simple strategy where we
make use of a primary classifier and a back-off classifier. For models
6 to 9, we augment each word with its corresponding 300 dimension word
embeddings vector. In the case of out-of-vocabulary words, we resort
to a back-off model that contains all features except the word
embeddings. For example, a model that makes use of words and word
embeddings, will make a prediction on the basis of the word features
alone. In case of the model that solely uses the embeddings (model 5),
the back-off classifier is represented by a majority vote classifier,
which classifies unseen words as inanimate.

\section{Results}\label{sec:results}

In Table \ref{tab:results-all} we present the results for all nine
models on the complete data set. For each model we report the
precision, recall, $F1$-score and Area Under the Precision-Recall
Curve (AUC_{PR}) for the animate words and the
inanimate words.
%AUC PR scores staan nu niet in de tabel, en worden schijnbaar niet gerapporteerd%
%positie van de tabellen is niet optimaal%
\begin{table}
\centering
\begin{tabular}{llrrr}
\toprule
features & class & Precision &  Recall & $F1$\\
\midrule
\multirow{2}{*}{embeddings}                            & inanimate     &       0.98 &    0.99 &    0.98 \\
                                                       & animate     &       0.93 &    0.89 &    0.91   \\
\multirow{2}{*}{word}                                  & inanimate     &       0.96 &    0.99 &    0.98 \\
                                                       & animate     &       0.94 &    0.78 &    0.85   \\
\multirow{2}{*}{word + embeddings}                     & inanimate     &       0.98 &    0.99 &    0.98 \\
                                                       & animate     &       0.94 &    0.90 &    0.91   \\
\multirow{2}{*}{word + PoS}                            & inanimate     &       0.97 &    0.99 &    0.98 \\
                                                       & animate     &       0.94 &    0.86 &    0.89   \\
\rowcolor{Gray}                                        & inanimate     &       0.98 &    0.99 &    0.99 \\
\rowcolor{Gray}\multirow{-2}{*}{word + PoS + embeddings}   & animate     &       0.94 &    0.91 &    0.93 \\
\multirow{2}{*}{word + PoS + lemma}                    & inanimate    &       0.97 &    0.99 &    0.98  \\
                                                       & animate     &       0.94 &    0.86 &    0.90   \\
\multirow{2}{*}{word + PoS + lemma + embeddings}       & inanimate     &       0.98 &    0.99 &    0.99 \\
                                                       & animate     &       0.94 &    0.91 &    0.93   \\
\multirow{2}{*}{word + PoS + lemma + dep}              & inanimate     &       0.97 &    0.99 &    0.98 \\
                                                       & animate     &       0.94 &    0.86 &    0.90   \\
\multirow{2}{*}{word + PoS + lemma + dep + embeddings} & inanimate     &       0.98 &    0.99 &    0.99 \\
                                                       & animate     &       0.94 &    0.92 &    0.93   \\
\bottomrule
\end{tabular}
\caption{Precision, Recall and $F1$-score for animate and inanimate classes per feature setting for all words.}
\label{tab:results-all}
\end{table}


All models perform well on classifying inanimate words. However, since
this is the majority class, it is of course far more interesting to
compare the performance of the models on the animate instances. It is
interesting to observe that the `simple' $n$-gram word model already
performs rather well. Adding more features, such as Part-of-Speech,
lemmata, etcetera, only has a positive impact on the recall of the
model, while leaving the precision untouched. As can be observed from
the table, employing the rather expensive dependency features shows
barely any improvement.

The model that only uses word embedding features is one of the best
performing models. This is a context insensitive model that operates
on the level of the vocabulary, which means that it will predict the
same outcome for each token of a particular word type. The high
precision and high recall show us that this model has acquired
knowledge about which words \emph{typically} group with animate
words and which with inanimate words. However, the models that combines
the word embeddings with the context sensitive features, such as word
$n$-grams or Part-of-Speech tags, outperform the context insensitive
model.

The best performance is achieved by the model that combines the word
features, part-of-speech tags and the word embeddings, which has an
$F1$-score of 0.93 on animate words and 0.99 on inanimate
words. Adding more features does not result in any more performance
gain.
%misschien hier dan nog een voorbeeld noemen, of verwijzen naar relevant cijfer in de tabel%
Table \ref{tab:results-noun} presents the results for all nouns and
names in the data set. The best performance is again achieved by the
model that combines the word features with the part-of-speech tags and
word embeddings, resulting in an $F1$-score of 0.92 for animate
instances and 0.95 for inanimate instances.
%moet in deze tabel ook AUC pr?%

\begin{table}
\centering
\begin{tabular}{llrrr}
\toprule
features & class &  Precision &  Recall & $F1$ \\
\midrule
\multirow{2}{*}{embeddings}                            & inanimate     &       0.90 &    0.96 &    0.92  \\
                                                       & animate       &       0.93 &    0.85 &    0.89  \\
\multirow{2}{*}{word}                                  & inanimate     &       0.78 &    0.98 &    0.87  \\
                                                       & animate       &       0.96 &    0.60 &    0.74  \\
\multirow{2}{*}{word + embeddings}                     & inanimate     &       0.90 &    0.97 &    0.93  \\
                                                       & animate       &       0.95 &    0.85 &    0.90  \\
\multirow{2}{*}{word + PoS}                            & inanimate     &       0.86 &    0.96 &    0.90  \\
                                                       & animate       &       0.93 &    0.78 &    0.84  \\
\rowcolor{Gray}                                        & inanimate     &       0.93 &    0.96 &    0.95  \\
\rowcolor{Gray}\multirow{-2}{*}{word + PoS + embeddings} & animate       &       0.95 &    0.90 &    0.92\\
\multirow{2}{*}{word + PoS + lemma}                    & inanimate     &       0.87 &    0.96 &    0.91  \\
                                                       & animate       &       0.94 &    0.80 &    0.86  \\
\multirow{2}{*}{word + PoS + lemma + embeddings}       & inanimate     &       0.93 &    0.96 &    0.94  \\
                                                       & animate       &       0.95 &    0.89 &    0.92  \\
\multirow{2}{*}{word + PoS + lemma + dep}              & inanimate     &       0.87 &    0.96 &    0.91  \\
                                                       & animate       &       0.93 &    0.80 &    0.86  \\
\multirow{2}{*}{word + PoS + lemma + dep + embeddings} & inanimate     &       0.93 &    0.96 &    0.95  \\
                                                       & animate       &       0.95 &    0.90 &    0.92  \\
\bottomrule
\end{tabular}
\caption{Precision, Recall, and $F1$ score for animate and inanimate classes
  per feature settings for all words tagged as noun.}
\label{tab:results-noun}
\end{table}


\section{Listing Characters in Stories}\label{sec:listing}

In a classical evaluation setup -- as with our approach -- general
practice is to train a computational system on a training data set of
which the performance is evaluated on a held-out test set. Even if we
include a separate development set, this setup could run the risk
of overfitting the data. Our annotated corpus contains a reasonably
diverse set of stories in terms of genre, yet it is fairly small and
stylistically rather homogeneous. Even though we performed a
cross-validation experiment, the chances of overfitting are
considerably high. In this section we investigate how well our animacy
detection model performs on a large collection of unannotated texts.
%ik zou misschien het stuk over evaluatie hier achteraan zetten: het is meer methodologie%
\subsection{Data}
For our evaluation we again make use of a collection of folk tales, consisting of a subsection of the Dutch Folktale Database\footnote{\url{http://www.verhalenbank.nl}}. The complete collection consists of about 42,000 folktales~\cite{meder:10}, and contains stories from various genres (e.g.~fairytales, legends, urban legends, jokes, personal narratives) in standard Dutch and Frisian, as well as a number of dialectal variants. Every entry in the database contains
meta-data about the story, including language, collector, place and
date of narration, keywords, names, and subgenre. For our paper we make use of a
subcollection comprising 16294 stories written in standard Dutch. The distribution of genres in the subcollection is the following: urban legends ($n=2795$), legends
($n=299$), jokes ($n=3986$), personal narratives ($n=693$), riddles
($n=1626$), sagas ($n=6045$) and fairy tales ($n=832$). We evaluate a
random sample of this subcollection ($n=212$) in which this genre
distribution is taken into account.

\subsection{Evaluation}
Our definition of animacy allows us to utilize our animacy detection
system to extract all characters from a story in a similar vain as
\cite{karsdorp:12}. In a real application, the system labels each noun
and name in a text for animacy. After removing duplicate words, this
produces a set of words that comprises the cast of a story. Without
golden standard annotations, however, we can only evaluate these
character sets for precision and not for recall. An alternative
approach is to produce a ranking of all words in a story where the
goal is to allocate the highest ranks to animate entities. This allows
us to evaluate the rankings using \emph{Average Precision} which
computes the average over precision scores at increasing points of
recall. Naturally, with this evaluation method, we still need to
manually evaluate the rankings. By using a rank cutoff and evaluating
a sample of all automatically annotated stories, we reduce the costly
manual labor to a minimum. We order all nouns and names in a story
using the output of the probabilistic decision function of the Maximum
Entropy classifier. After removing duplicate words, this produces a
final ranking. The rankings are evaluated using \emph{Average
  Precision} with a rank cutoff at 50.

\subsection{Results}

\begin{figure}
\centering
\includegraphics[width=\textwidth]{precision-recall-curve}
\caption{Precision-Recall Curves and Average Precision scores per
genre.}
\label{fig:prec-recall-curve}
\end{figure}

We present the results in Figure \ref{fig:prec-recall-curve} in which
we show the Precision-Recall curve as well as the Average Precision
score for each genre. The system performs well, especially on fairy
tales (AP=0.97) and jokes (AP=0.94). The lowest performance is
measured on riddles (AP=0.85). This lower score is mainly the result
of the system's inability to position the word \emph{blondje}
(`dumb blonde' with a pejorative connotation, which is prevalent in the so-called `dumb blonde jokes') high up the ranking. 

%misschien nog verschillende scores benoemen?%
%misschien ook nog iets zeggen over de karakter ranking? Of was dat juist het onderdeel dat we er niet meer in gingen betrekken?%
\begin{sidewaysfigure}
\includegraphics[width=\linewidth]{map}
\caption{Visualization of characters in the Dutch Folktale Database based on their embeddings using t-SNE.}
\label{fig:map}
\end{sidewaysfigure}

\section{Concluding Remarks}

The approach taken in this paper to create a model for animacy
classification using linguistically uninformed features proves to be
successful. We compared the performance of linguistically informed
models (using features such as part-of-speech and dependency tags) to
models that make use of lower dimensional representations of the
data. With the exception of the model that solely makes use of these
representations, all models benefit from adding these features. The
model that requires the least linguistic information (word $n$-grams
plus word embeddings) outperforms all linguistically informed
models. The best results are reported by the model that combines word
$n$-grams with Part-of-Speech $n$-grams and word embeddings.

We have the following recommendation for future research. Natural
language processing models such as co-reference resolution or
linguistic parsing could benefit from a module that filters animate
from inanimate candidate words. Since these models typically depend
heavily on linguistic features, it is important that additional
features, such as animacy, are not dependent on these features as
well. Our linguistically uninformed model for animacy detection
provides such a module.


\subparagraph*{Acknowledgments}

The work on which this paper is based has been supported by the
Computational Humanities Programme of the Royal Netherlands Academy of
Arts and Sciences, under the auspices of the Tunes \& Tales
project. For further information, see \url{http://ehumanities.nl}.

\bibliography{paper}

\end{document}
%%% Local Variables:
%%% mode: latex
%%% TeX-master: t
%%% End:
