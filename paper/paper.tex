\documentclass[a4paper,UKenglish]{oasics}

\usepackage{microtype, graphicx, booktabs, multirow, color, colortbl}

\usepackage{blindtext}
\bibliographystyle{plain}

\definecolor{Gray}{gray}{0.9}
\usepackage[utf8]{inputenc}

\title{Animacy Detection in Stories}
\author[1]{Folgert Karsdorp}
\author[2]{Marten van der Meulen}
\author[3]{Theo Meder}
\author[4]{Antal van den Bosch}
\affil[1]{Meertens Institute\\
  Amsterdam, The Netherlands\\
  \texttt{folgert.karsdorp@meertens.knaw.nl}}
\affil[2]{Meertens Institute\\
  Amsterdam, The Netherlands\\
  \texttt{marten.van.der.meulen@meertens.knaw.nl}}
\affil[3]{Meertens Institute\\
  Amsterdam, The Netherlands\\
  \texttt{theo.meder@meertens.knaw.nl}}
\affil[4]{Radboud University\\
  Nijmegen, The Netherlands\\
  \texttt{a.vandenbosch@let.ru.nl}}
%\authorrunning{F. Karsdorp and M. van der Meulen and A. van den Bosch}
%\Copyright{Folgert Karsdorp and Marten van der Meulen and Antal van
%  den Bosch}
\authorrunning{F. Karsdorp and M. van der Meulen and T. Meder and A. van den Bosch}
\Copyright{Folgert Karsdorp and Marten van der Meulen and Theo Meder
  and Antal van den Bosch}
\subjclass{I.2.7 Natural Language Processing}
\keywords{animacy detection, deep learning, folklore}

%Editor-only macros:: begin (do not touch as author)%%%%%%%%%%%%%%%%%%%%%%%%%%%%%%%%%%
\serieslogo{}%please provide filename (without suffix)
\volumeinfo%(easychair interface)
  {Billy Editor and Bill Editors}% editors
  {2}% number of editors: 1, 2, ....
  {Conference/workshop/symposium title on which this volume is based on}% event
  {1}% volume
  {1}% issue
  {1}% starting page number
\EventShortName{}
\DOI{10.4230/OASIcs.xxx.yyy.p}% to be completed by the volume editor
% Editor-only macros::end %%%%%%%%%%%%%%%%%%%%%%%%%%%%%%%%%%%%%%%%%%%%%%%


\begin{document}
\maketitle

\begin{abstract}
  This paper presents a linguistically uninformed computational model
  for animacy classification. The model makes use of word $n$-grams in
  combinations with lower dimensional word embedding representations
  that were learned from a web-scale corpus. We compare the model to a
  number of linguistically informed models that use features such as
  dependency tags and show competitive results.
\end{abstract}

\section{Introduction}

For almost all species in the world, the capacity to distinguish
animate objects from inanimate objects is essential to their
survival. Those objects could be prey, for example, or predators or
mates. The fundamental nature of the distinction between animate and
inanimate appears in the early acquisition of the division
\cite{opfer:02}. Children of less than 6 months old are well able to
make the division. Recent brain research shows that the distinction
appears in the organization of the brain (e.g.~\cite{gao:12}). For some
researchers this provides evidence for the idea that the division
between animate and inanimate is an innate part of how we see the
world.

A common view supported by philosophers such as Daniel Dennett is that
animacy and aliveness are better to be treated as epistemological
stances rather than fixed states in the world \cite{dennett:96}. Not
ineffable qualia but behavioral capacity defines our stance towards
objects. Depending on whether people \textit{think} that an object is
animate, they utilize different cognitive strategies to explain and
predict the actions of those objects. The following opening of a
well-known story will make this point more clear:

\begin{quotation} {\it A farmer bought a pancake on the market. Once
    he got home, the farmer was hungry and began to bake the
    pancake. The farmer tried one of his skillful flipping techniques,
    but he failed and the pancake fell on the ground. Coincidentally,
    the door of the kitchen was open and the pancake rolled out to the
    field, as hard as he could\ldots}
\end{quotation}

As readers (or listeners) of a story, we choose to view participating
objects at varying levels of abstraction in order to predict their
behavior. Dennett \cite{dennett:96} defines three levels of
abstraction: (1) the \textit{physical stance}, (2) the \textit{design
  stance} and (3) the \textit{intentional stance}. The physical stance
deals with predictions about objects given their physical
properties. The design stance is slightly more abstract and deals with
concepts like purpose, function or design. Finally the intentional
stance is concerned with belief, thinking and intentions. These are
all cognitive strategies we use to predict and explain the actions of
objects in our environment. Interestingly, in the process of reading
the opening of the story about the fleeing pancake, readers and
listeners experience the transition from one strategy to the next
quite clearly. Initially, the pancake is interpreted from a physical
stance or perhaps the more abstract design stance in terms of the
purpose (i.e. to stave off hunger). It is only at the last adverbial
phrase `as hard as he could' that we start to wonder whether we should
adopt to the yet more abstract intentional stance and consider the
pancake to be a rational agent.

Given the fundamental nature of the distinction between animate and
inanimate, it is perhaps not too surprising that it has proven to be
useful in a variety of natural language processing tasks dealing with
e.g.\ anaphora resolution and dependency parsing
\cite{orasan:07,lee:13,ovr:niv:07}. Existing methods to automatically
label natural language for animacy are usually rule based or apply
machine learning techniques or make a combination of both. Common to
most approaches is that they make use of semantic lexicons with
information about animacy and syntactic cues in a text. Both feature
types are relatively costly to obtain as they require large
vocabularies or syntactic parsing systems, which are besides for a
handful of western languages, often not available.

In this paper we present a new linguistically uninformed model to
automatically label texts for animacy. We will show that we can (mostly)
do away with features that require syntactic parsing or semantic
lexicons while still yielding competitive performance. We will focus
on labeling animacy in stories because stories pose some interesting
problems to automatic systems of animacy recognition. As the example
of the fleeing pancake already made clear, quite some entities in
stories, though inanimate in the `real' world, exhibit all behavioral
capacities to consider them to be animate and rational agents. Such
examples make a clear case for developing dynamic, data driven systems
that do not rely too much on static and fixed world knowledge.

The rest of this paper is structured as follows. We will start with an
short overview of existing techniques to automatically label texts for
animacy (\S\ref{sec:previous-work}). After a description of the corpus
used in our study and how the annotations of the corpus have been
established (\S\ref{sec:data}), we will give an account of our
computational models in Section \ref{sec:models}. We report on the
empirical results in Section \ref{sec:results}. The last section
offers our conclusions and possible directions for future research.


\section{Previous Work}\label{sec:previous-work}


In most of the literature on automatic animacy detection, people make
a binary distinction between animate and inanimate. \cite{bowman:12}
for example, define objects to be animate if they are alive and have
the ability to move under their own will. \cite{orasan:07} define
animacy in the context of anaphora resolution: ``if its referent can
also be referred to using one of the pronouns he, she, him, her, his,
hers, himself, herself, or a combination of such pronouns
(e.g. his/her)''. Not only are some parts of these definitions somewhat
circular, we believe that they are both essentialistic and not
particularly suited for the rich and wondrous entities that live in
the realm of stories. We choose to define animacy in terms of
Dennett's intentional stance which comes down to the question whether
``you decide to treat the object whose behavior is to be predicted as
a rational agent'' \cite[pp. 17]{dennett:96}.

A handful of papers deal with automatic animacy detection. Most
approaches make use of rule-based systems or machine learning systems
with morphological and syntactic features. \cite{evans:00} present a
rule-based system that makes use of the lexical-semantic database
WordNet. They label each synset in WordNet for animacy. Using a
variety of rules to detect the head of an NP, they use the fraction of
synsets in which a particular noun occurs to arrive at a
classification for animacy. \cite{orasan:01} extend their previous
algorithm by first determining the animacy of senses from WordNet on
the basis of an annotated corpus. They then apply a $k$-nearest
neighbor classifier using a number of lexical and syntactic features
and features derived from WordNet to arrive at a final animacy
classification.

\cite{ovrelid:04,ovrelid:05,ovrelid:06,ovrelid:08,ovrelid:09} present
a number of animacy classifiers that make use of syntactic and
morphological features. Some of the features used are: how often has a
noun been analyzed as `subject' or `object'? How often does a noun
occur in a passive by-phrase? How often does a noun occur as a
subject, immediately followed by animate personal pronouns or
inanimate personal pronouns?  These features are then aggregated for
each lemma after which a machine learning system (decision tree or
$k$-nearest neighbor classifier) is trained. In \cite{bowman:12} a
similar approach is presented. In this study a maximum entropy
classifier is trained on the basis of three feature types: (1)
bag-of-words with and without their corresponding Part-of-Speech tags,
(2) internal syntactic features such as the syntactic head and (3)
external syntactic features that describe the dependency relation of a
noun to a verb (i.e.\ subject relation, object relation etc.)  This is
the only study that makes use of a fully labeled corpus for animacy.
Not so much concerned with animacy detection, \cite{karsdorp:12}
attempt to extract the cast (i.e.\ all characters) from a
story. Similar to \cite{bowman:12} they rely on dependency tags to
extract the subjects of direct and indirect speech.

\cite{bloem:13} present a model that attempts to generalize the
animacy information in a lexical-semantic database of Dutch by
augmenting `non-ambiguous' animate entries with contextual information
from a large treebank of Dutch. They apply a $k$-nearest neighbor
algorithm with distributional lexical features that aim to capture the
association using measures like mutual information between a verb or
adjective and a particular noun. The idea is that nouns that occur in
similar contexts as animate nouns are more likely to be animate than
nouns that occur more frequently in contexts similar to inanimate nouns.

\cite{moore:13} present an interesting approach that combines a number
of animacy classifiers in a voting scheme and aims at an interpretable
and correctable model of animacy classification. They combine a
variety of classifiers such as the WordNet-based approach of
\cite{evans:00}, dictionary sources and systems for named entity
recognition.

% commentary on above approaches
There are a number of problems with the approaches mentioned
above. First, almost all of them heavily rely on costly,
linguistically informed features derived from lexical-semantic
databases or syntactic parsing. For most languages in the world,
however, we cannot rely on these resources. Second, animacy detection
is often seen as a useful feature for a range of natural language
processing techniques, such as anaphora resolution and syntactic
parsing. The mutual dependence of these techniques and animacy
detection, however, leads to a chicken and egg situation. Another
problem with the approaches above is that they are lemma based which
means that the models are generally insensitive to different usages of
a particular word. A pancake is typically not an animate entity, but,
as we have seen, whether it is animate or not is only to be determined
in relation to the story in which it appears. We therefore need to
develop a dynamic, data driven system that is token based and does not
rely too much on static and fixed world knowledge.

\section{The Data, Annotation and Preprocessing}\label{sec:data}

Our corpus consists of 74 Dutch stories from the collection
\textit{Volkssprookjes uit Nederland en Vlaanderen} compiled by
\cite{sinninghe:78}. The collection contains retellings of popular
stories from the Netherlands and Flanders, such as \textit{The Bremen
  Town Musicians} (ATU 130) and \textit{The Table, the Ass, and the
  Stick } (ATU 563). Some of the less well-know stories are
\textit{Cock, Hen, Duck, Pin, and Needle on a Journey} (AT 0210) and
(ATU 780) \textit{The Singing Bone}. A digital version of the
collection is available in the Dutch Folktale Database from the
Meertens Institute (corpus SINVSUNV.20E).\footnote{See
  \url{http://www.verhalenbank.nl}} We chose this collection because
of its homogeneity while it still displays enough diversity: (1) all
tales are written in standard Dutch, (2) the stories are edited by the
same editor, (3) the tales have a comparable length and (4) the
collection contains tales from a number of different genres, such as
fairy tales, legends, riddles etc.

We assigned to each word in the collection a binary label `animate' or
`inanimate'. 11542 words were labeled as animate and 62926 words as
inanimate. The vocabulary size is 5549 with 743 unique animate words
and 5011 unique inanimate words. The annotations consist of 4627
animate words that have been tagged as noun or name against 6878
inanimate words.

After tokenization with the tokenization module of the Python software
package Pattern \cite{smedt:12} we fed all stories to the
state of the art syntactic parser for Dutch, Alpino \cite{bouma:01}.

\section{Experimental Setup}\label{sec:models}
This section describes our experimental setup including the features
used, the machine learning models we applied and our methods of
evaluation.\footnote{The data set and the code to perform the
  experiments are made available online at
  \url{https://fbkarsdorp.github.io/animacy-detection}}.

\subsection{Task description}

We formulate the problem of animacy detection as a classification
problem where the goal is to assign to each word in a document a label
indicating whether it represents an animate or an inanimate word.

\subsection{Evaluation}

Animate words are far outnumbered by inanimate words in our
collection. Reporting accuracy scores would therefore provide skewed
results too much in favor of the majority category. The rarity of
animate words makes evaluation measures such as the well-known $F1$-score
more appropriate. For all experiments we report on the precision,
recall and $F1$-score \cite{rijsbergen:79}. In most of the literature on
animacy detection results are only given for the classification of
nouns or noun phrases. We will report the results for all words in a
text and on those specifically for nouns separately.

In real world applications, an animacy detection system will most
likely be faced with completely new texts instead of single words. It
is therefore important to construct a training and test procedure in
such a way that it mimics this situation as closely as possible. If we
would make a random split of say 80\% of the data for training and
20\% for testing on the word level, we run the risk of mixing training
data with test data making it too easy for a system to rely on words
it has seen from the same text. \cite{bowman:12} fall into this trap
by making a random split in their data on the sentence level. It is
very likely that in their setup sentences from the same document are
present in both the training data and the test data which makes their
evaluation unrealistic. To circumvent this problem, we train our model
on 80\% of the stories (n=59) and test on the remaining 15 stories.

\subsection{Features}

We explore a range of different features and feature combinations
including lexical features, morphological features, syntactic features
and semantic features.

\subsubsection{Lexical features}
We take a sliding window approach where for each focus word (i.e.\ the
word for which we want to predict whether it is animate or not) we
extract $n$ words to the left, $n$ words to the right and the focus
word itself. In all experiments we set $n$ to 3. In addition to the
word forms, for each word in a window we extract its lemma.

\subsubsection{Morphological Features}
For each word we extract its Part-of-Speech tag. For reasons of
comparability we choose to use the tags as provided by the output of
the syntactic parser Alpino, instead of a more specialized
Part-of-Speech tagger. Again, we take a sliding window approach and
extract the Part-of-Speech tags for three word left of the focus word,
three word right of the focus word and the tag of the focus word
itself.

\subsubsection{Syntactic Features}
We extract the dependency tag for each word and its $n$ neighbors to
the right and to the left as provided by the syntactic parser
Alpino. Animate entities tend to take the position of subject or
object in a sentence which is why this feature is expected and has
proven to perform rather well.

\subsubsection{Semantic Features} The most innovative feature we have
included in our model is concerned with semantic similarity. In his
\textit{Philophische Untersuchungen} Wittgenstein already suggests
that ``Die Bedeutung eines Wortes ist sein Gebrauch in der Sprache''
(PI 43). The well-known insight in computational linguistics that the
meaning of words can be approximated by comparing the linguistic
contexts in which words appear, is an implementation of this very
idea. The idea is very simple, and equally strong: words that often
co-appear with the same set of words, will have a more similar
meaning. Recently, there has been a lot of interest in procedures that
can automatically induce so-called `word-embeddings' from large,
unannotated collections of texts. These models typically attempt to
learn a lower-dimensional vector for each word in the vocabulary which
captures the typical co-occurrence patterns of the word in the
corpus. The similarity between words can then be approximated by
applying arithmetics to these vectors of word embeddings.

We have trained word embeddings with 300 dimensions using the popular
skip-gram architecture \cite{mikolov:13} on the Dutch corpus of COW
(COrpora from the Web).\footnote{The word embedding vectors have been
  published online at XXX.} COW is a collection of linguistically
processed web corpora for English, Dutch, Spanish, French, Swedish and
German \cite{schaefer:12}. The 2014 Dutch corpus contains 6.8 billion
word tokens. The idea behind using the word embeddings is that
similarities between animate words can be estimated by inspecting the
context in which they occur. The hypothesis is that the word
embeddings of an animate word are more similar to those of other
animate words than than to those of inanimate words. To give an
example, the 5 nearest neighbors of \textit{Assepoester}
(`Cinderella') are: \textit{Sneeuwwitje} (`Snow White') (sim = 0.807),
\textit{Roodkapje} (`Little Red Riding Hood') (sim = 0.747),
\textit{Doornroosje} (`Sleeping Beauty') (sim = 0.715),
\textit{Repelsteeltje} (`Rumpelstiltskin') (sim = 0.669) and
\textit{Alladin} (0.645). In contrast, the 5 most similar words to
\textit{boek} (`book') are: \textit{boekje} (`book-\textsc{dim}')
(sim=0.8), \textit{kinderboek} (`children's book') (0.757),
\textit{stripboek} (`comic') (0.713), \textit{kortverhaal} (`short
story') (0.709), \textit{jeugdboek} (`youth book') (0.706).

We add the word embedding vectors to our animacy model as follows. We
augment each word in the data with its corresponding 300 dimension
word embeddings vector. Out-of-vocabulary words were given a vector were all
dimensions are set to zero.


\subsection{Models}
We employ a Maximum Entropy classifier with L2 regularization as
implemented in \cite{sklearn}. In all experiments, we set the
regularization strength parameter $C$ to 1.

We compare nine models in which we make use of different feature
combinations: (1) words, (2) words and Part-of-Speech tags, (3)
words, Part-of-Speech tags and lemmata, (4) words, Part-of-Speech
tags, lemmata and dependency tags, (5) word embeddings and (6-9) the
features in model 1 to 4 plus word embeddings.


\section{Results}\label{sec:results}

\begin{table}
\centering
\begin{tabular}{llrrrr}
\toprule
features                                                 &           & precision & recall & $F1$ & avg $F1$              \\ \midrule
\multirow{2}{*}{word}                                    & animate   & 0.93      & 0.80   & 0.86 & \multirow{2}{*}{0.96} \\
                                                         & inanimate & 0.96      & 0.99   & 0.98 &                       \\
\multirow{2}{*}{word + PoS}                              & animate   & 0.92      & 0.88   & 0.90 & \multirow{2}{*}{0.97} \\
                                                         & inanimate & 0.98      & 0.99   & 0.98 &                       \\
\multirow{2}{*}{word + PoS + lemma}                      & animate   & 0.93      & 0.89   & 0.91 & \multirow{2}{*}{0.97} \\
                                                         & inanimate & 0.98      & 0.99   & 0.98 &                       \\
\multirow{2}{*}{word + PoS + lemma + dep}                & animate   & 0.93      & 0.89   & 0.91 & \multirow{2}{*}{0.97} \\
                                                         & inanimate & 0.98      & 0.99   & 0.98 &                       \\
\multirow{2}{*}{word + embeddings}                       & animate   & 0.92      & 0.93   & 0.92 & \multirow{2}{*}{0.98} \\
                                                         & inanimate & 0.99      & 0.99   & 0.99 &                       \\
\rowcolor{Gray}                                          & animate   & 0.92      & 0.93   & 0.93 &                       \\
\rowcolor{Gray}\multirow{-2}{*}{word + PoS + embeddings} & inanimate & 0.99      & 0.99   & 0.99 & \multirow{-2}{*}{0.98}\\
\multirow{2}{*}{word + PoS + lemma + embeddings}         & animate   & 0.93      & 0.93   & 0.93 & \multirow{2}{*}{0.98} \\
                                                         & inanimate & 0.99      & 0.99   & 0.99 &                       \\
\multirow{2}{*}{word + PoS + lemma + dep + embeddings}   & animate   & 0.92      & 0.93   & 0.93 & \multirow{2}{*}{0.98} \\
                                                         & inanimate & 0.99      & 0.99   & 0.99 &                       \\
\multirow{2}{*}{word embeddings}                         & animate   & 0.90      & 0.92   & 0.91 & \multirow{2}{*}{0.97} \\
                                                         & inanimate & 0.99      & 0.98   & 0.98 &                       \\

\bottomrule
\end{tabular}
\caption{Precision, recall and $F1$ scores for animate and inanimate classes per feature settings for all words.}
\label{tab:results-all}
\end{table}


In table \ref{tab:results-all} we present the results for all nine
models on the complete data set. For each model we report the
precision, recall and $F1$-score for the animate words and the
inanimate words. All models perform well on classifying inanimate
words. However, since this is the majority class, it is far more
interesting to compare the performance of the models on the animate
instances. It is interesting to observe that the `simple' $n$-gram
word model already performs rather well. Adding more features, like
Part-of-Speech, lemmata, etcetera, only has a positive impact on the
recall of the model leaving the precision untouched. As can be
observed from the table, the rather expensive dependency features show
barely any improvement. The model that only uses word embedding
features, is one of the best performing models. This was slightly
unexpected, since this is the model in which features are context
insensitive. However, the model that combines the word embeddings with
the word $n$-gram features outperforms the other linguistically
informed models that make use of lemmata, Part-of-Speech tags or
dependency information. The best performance is achieved by the model
that combines the word features, Part-of-Speech tags and the word
embeddings with an $F1$-score of 0.93 for animate words and 0.99 on
inanimate words. Adding more features does not result in any more
performance gain.


\begin{table}
\centering
\begin{tabular}{llrrrr}
\toprule
features                                               &           & precision & recall & $F1$ & avg $F1$              \\ \midrule
\multirow{2}{*}{word}                                  & animate   & 0.96      & 0.64   & 0.77 & \multirow{2}{*}{0.84} \\
                                                       & inanimate & 0.81      & 0.98   & 0.89 &                       \\
\multirow{2}{*}{word + PoS}                            & animate   & 0.92      & 0.80   & 0.86 & \multirow{2}{*}{0.90} \\
                                                       & inanimate & 0.89      & 0.95   & 0.92 &                       \\
\multirow{2}{*}{word + PoS + lemma}                    & animate   & 0.93      & 0.82   & 0.87 & \multirow{2}{*}{0.90} \\
                                                       & inanimate & 0.89      & 0.96   & 0.92 &                       \\
\multirow{2}{*}{word + PoS + lemma + dep}              & animate   & 0.93      & 0.82   & 0.87 & \multirow{2}{*}{0.91} \\
                                                       & inanimate & 0.90      & 0.96   & 0.93 &                       \\
\multirow{2}{*}{word + embeddings}                     & animate   & 0.92      & 0.91   & 0.92 & \multirow{2}{*}{0.94} \\
                                                       & inanimate & 0.95      & 0.95   & 0.95 &                       \\
\rowcolor{Gray}                                        & animate   & 0.92      & 0.92   & 0.92 &                       \\
\rowcolor{Gray}\multirow{-2}{*}{word + PoS + embeddings} & inanimate & 0.95    & 0.95   & 0.95 & \multirow{-2}{*}{0.94}\\
\multirow{2}{*}{word + PoS + lemma + embeddings}       & animate   & 0.92      & 0.91   & 0.92 & \multirow{2}{*}{0.94} \\
                                                       & inanimate & 0.95      & 0.95   & 0.95 &                       \\
\multirow{2}{*}{word + PoS + lemma + dep + embeddings} & animate   & 0.92      & 0.91   & 0.92 & \multirow{2}{*}{0.94} \\
                                                       & inanimate & 0.94      & 0.95   & 0.95 &                       \\
\multirow{2}{*}{word embeddings}                       & animate   & 0.88      & 0.90   & 0.89 & \multirow{2}{*}{0.92} \\
                                                       & inanimate & 0.94      & 0.92   & 0.93 &                       \\

\bottomrule
\end{tabular}
\caption{Precision, recall and $F1$ scores for animate and inanimate classes
  per feature settings for all words tagged as noun.}
\label{tab:results-noun}
\end{table}

Table \ref{tab:results-noun} presents the results for all nouns and
names in the data set. The best performance is again achieved by the
model that combines the word features with the Part-of-Speech tags and
word embeddings, resulting in an $F1$-score of 0.92 for animate
instances and 0.95 for inanimate instances.


\section{Concluding Remarks}

The approach taken in this paper to create a model for animacy
classification using linguistically uninformed features proves to be
successful. We compared the performance of linguistically informed
models (using features such as Part-of-Speech and dependency tags) to
models that make use of lower dimensional representations of the
data. With the exception of the model that solely makes use of these
representations, all models benefit from adding these features. The
model that requires the least linguistic information (word $n$-grams
plus word embeddings) outperforms all linguistically informed
models. The best results are reported by the model that combines word
$n$-grams with Part-of-Speech $n$-grams and word embeddings.

We have the following recommendation for future research. Natural
language processing models such as co-reference resolution or
linguistic parsing could benefit from a module that filters animate
from inanimate candidate words. Since these models typically depend
heavily on linguistic features, it is important that additional
features, such as animacy, are not dependent on these features as
well. Our linguistically uninformed model for animacy detection
provides such a module.



\subparagraph*{Acknowledgments}

The work on which this paper is based has been supported by the
Computational Humanities Programme of the Royal Netherlands Academy of
Arts and Sciences, under the auspices of the Tunes \& Tales
project. For further information, see \url{http://ehumanities.nl}.

\bibliography{paper}

\end{document}
%%% Local Variables:
%%% mode: latex
%%% TeX-master: t
%%% End:
