\documentclass[a4paper,UKenglish]{oasics}

\usepackage{microtype, graphicx, booktabs, multirow, color, colortbl}

\usepackage{gb4e}
\bibliographystyle{plain}

\definecolor{Gray}{gray}{0.9}
\usepackage[utf8]{inputenc}

\title{Animacy Detection in Stories}
\author[1]{Folgert Karsdorp}
\author[2]{Marten van der Meulen}
\author[3]{Theo Meder}
\author[4]{Antal van den Bosch}
\affil[1]{Meertens Institute\\
  Amsterdam, The Netherlands\\
  \texttt{folgert.karsdorp@meertens.knaw.nl}}
\affil[2]{Meertens Institute\\
  Amsterdam, The Netherlands\\
  \texttt{marten.van.der.meulen@meertens.knaw.nl}}
\affil[3]{Meertens Institute\\
  Amsterdam, The Netherlands\\
  \texttt{theo.meder@meertens.knaw.nl}}
\affil[4]{Radboud University\\
  Nijmegen, The Netherlands\\
  \texttt{a.vandenbosch@let.ru.nl}}
%\authorrunning{F. Karsdorp and M. van der Meulen and A. van den Bosch}
%\Copyright{Folgert Karsdorp and Marten van der Meulen and Antal van
%  den Bosch}
\authorrunning{F. Karsdorp and M. van der Meulen and T. Meder and A. van den Bosch}
\Copyright{Folgert Karsdorp and Marten van der Meulen and Theo Meder
  and Antal van den Bosch}
\subjclass{I.2.7 Natural Language Processing}
\keywords{animacy detection, deep learning, folklore}

%Editor-only macros:: begin (do not touch as author)%%%%%%%%%%%%%%%%%%%%%%%%%%%%%%%%%%
\serieslogo{}%please provide filename (without suffix)
\volumeinfo%(easychair interface)
  {Billy Editor and Bill Editors}% editors
  {2}% number of editors: 1, 2, ....
  {Conference/workshop/symposium title on which this volume is based on}% event
  {1}% volume
  {1}% issue
  {1}% starting page number
\EventShortName{}
\DOI{10.4230/OASIcs.xxx.yyy.p}% to be completed by the volume editor
% Editor-only macros::end %%%%%%%%%%%%%%%%%%%%%%%%%%%%%%%%%%%%%%%%%%%%%%%


\begin{document}
\maketitle

\begin{abstract}
  This paper presents a linguistically uninformed computational model
  for animacy classification. The model makes use of word $n$-grams in
  combinations with lower dimensional word embedding representations
  that were learned from a web-scale corpus. We compare the model to a
  number of linguistically informed models that use features such as
  dependency tags and show competitive results.
\end{abstract}

\section{Introduction}

For almost all species in the world, the capacity to distinguish
animate objects from inanimate objects is essential to their
survival. Those objects could be prey, for example, or predators or
mates. The fundamental nature of the distinction between animate and
inanimate is reflected in the fact that the division is acquired very
early in life~\cite{opfer:02}. Children of less than 6 months old
are well able to make the division. Moreover, recent brain research
shows that the distinction appears in the organization of the brain
(e.g.~\cite{gao:12}). For some researchers this provides evidence for
the idea that the division between animate and inanimate is an innate
part of how we see the world.

A common view supported by philosophers such as Daniel Dennett is that
animacy and aliveness are to be treated as epistemological
stances rather than fixed states in the world \cite{dennett:96}: not
ineffable qualia but behavioral capacity defines our stance towards
objects. Depending on whether people \textit{think} that an object is
animate, they utilize different cognitive strategies to explain and
predict the actions of those objects. Th following opening of a
well-known story will make this point more clear:

\begin{quotation} {\it A farmer bought a pancake on the market. Once
    he got home, the farmer was hungry and began to bake the
    pancake. The farmer tried one of his skillful flipping techniques,
    but he failed and the pancake fell on the ground. Coincidentally,
    the door of the kitchen was open and the pancake rolled out to the
    field, as hard as he could\ldots}
\end{quotation}

As readers (or listeners) of a story, we choose to view participating
objects at varying levels of abstraction in order to predict their
behavior. Dennett \cite{dennett:96} defines three levels of
abstraction: (1) the \textit{physical stance}, (2) the \textit{design
  stance} and (3) the \textit{intentional stance}. The physical stance
deals with predictions about objects given their physical
properties. The design stance is slightly more abstract and deals with
concepts like purpose, function or design. Finally the intentional
stance is concerned with belief, thinking and intentions. These are
all cognitive strategies we use to predict and explain the actions of
objects in our environment. Interestingly, in the process of reading
the opening of the story about the fleeing pancake, readers and
listeners experience the transition from one strategy to the next
quite clearly. Initially, the pancake is interpreted from a physical
stance or perhaps the more abstract design stance in terms of the
purpose (i.e. to stave off hunger). It is only at the last adverbial
phrase `as hard as he could' that we start to wonder whether we should
adopt to the yet more abstract intentional stance and consider the
pancake to be a rational agent.

Given the fundamental nature of the distinction between animate and
inanimate, it is perhaps not too surprising that it has proven to be
useful in a variety of natural language processing tasks dealing with
e.g.\ anaphora resolution and dependency parsing
\cite{orasan:07,lee:13,ovr:niv:07}. Existing methods for the automatic
labeling of text for animacy are usually either rule based or apply
machine learning techniques, or make a combination of both. Common to
most approaches is that they make use of semantic lexicons with
information about animacy and syntactic cues in a text. Both feature
types are relatively costly to obtain as they require large
vocabularies or syntactic parsing systems, which, with the exception
of a few languages, are not readily available.

In this paper we present a new linguistically uninformed model to
automatically label texts for animacy. We show that we can (mostly)
do away with features that require syntactic parsing or semantic
lexicons while still yielding competitive performance. We focus
on labeling animacy in stories because stories pose some interesting
problems to automatic systems of animacy recognition. As the example
of the fleeing pancake already made clear, quite some entities in
stories, though inanimate in the `real' world, exhibit all behavioral
capacities to consider them animate and rational agents. Such
examples make a clear case for developing dynamic, data driven systems
that do not rely too much on static and fixed world knowledge.

The rest of this paper is structured as follows. We will start with a
short overview of existing techniques for automatically labeling
animacy in texts (\S\ref{sec:previous-work}). After a description of
the corpus used in our study and how the annotations of the corpus
have been established (\S\ref{sec:data}), we will give an account of
our computational models in Section \ref{sec:models}. We report on the
empirical results in Section \ref{sec:results}. The last section
offers our conclusions and possible directions for future research.


\section{Previous Work}\label{sec:previous-work}

%ik wil eigenlijk meer uitleg in deze paragraaf:
In most of the literature on automatic animacy detection, a static,
binary distinction is made between animate and
inanimate. \cite{bowman:12} for example, define objects as animate if
they are alive and have the ability to move under their own
will. \cite{orasan:07} define animacy in the context of anaphora
resolution: something is animate ``if its referent can also be
referred to using one of the pronouns he, she, him, her, his, hers,
himself, herself, or a combination of such pronouns
(e.g. his/her)''. However, some parts of these definitions are
somewhat circular and essentialistic. More importantly, they are not
particularly applicable to the rich and wondrous entities that live in
the realm of stories. As our example above showed, it is possible for
an otherwise as inanimate labeled object to attain animacy. We
therefore choose to define animacy in terms of Dennett's intentional
stance, which is more dynamic, and which ultimately comes down to the
question whether ``you decide to treat the object whose behavior is to
be predicted as a rational agent'' \cite[pp. 17]{dennett:96}.

A handful of papers deal with automatic animacy detection. Most
approaches make use of rule-based systems or machine learning systems
with morphological and syntactic features. \cite{evans:00} present a
rule-based system that makes use of the lexical-semantic database
WordNet. They label each synset in WordNet for animacy. Using a
variety of rules to detect the head of an NP, they use the fraction of
synsets in which a particular noun occurs to arrive at a
classification for animacy. \cite{orasan:01} extend their previous
algorithm by first determining the animacy of senses from WordNet on
the basis of an annotated corpus. They then apply a $k$-nearest
neighbor classifier using a number of lexical and syntactic features
and features derived from WordNet to arrive at a final animacy
classification.

\cite{ovrelid:04,ovrelid:05,ovrelid:06,ovrelid:08,ovrelid:09} present
a number of animacy classifiers that make use of syntactic and
morphological features. These features include the frequency of
analysis of the noun as `subject' or `object', the frequency of the
occurence of a noun in a passive $by$-phrase, and the frequency of the
noun as a subject followed by either animate personal pronouns or
inanimate personal pronouns. These features are then aggregated for
each lemma after which a machine learning system (decision tree or
$k$-nearest neighbor classifier) is trained. In \cite{bowman:12} a
similar approach is presented. In this study a Maximum Entropy
classifier is trained on the basis of three feature types: (1)
bag-of-words with and without their corresponding Part-of-Speech tags,
(2) internal syntactic features such as the syntactic head and (3)
external syntactic features that describe the dependency relation of a
noun to a verb (i.e.\ subject relation, object relation etc.)  This is
the only study that makes use of a fully labeled corpus for animacy.
Partially related to animacy detection, \cite{karsdorp:12}
attempt to extract the cast (i.e.\ all characters) from a
story. Similar to \cite{bowman:12} they rely on dependency tags to
extract the subjects of direct and indirect speech.

\cite{bloem:13} present a model that attempts to generalize the
animacy information in a lexical-semantic database of Dutch by
augmenting `non-ambiguous' animate entries with contextual information
from a large treebank of Dutch. They apply a $k$-nearest neighbor
algorithm with distributional lexical features that aim to capture the
association using measures like mutual information between a verb or
adjective and a particular noun. The idea is that nouns that occur in
similar contexts as animate nouns are more likely to be animate than
nouns that occur more frequently in contexts similar to inanimate nouns.

\cite{moore:13} present an approach that combines a number of animacy
classifiers in a voting scheme and aims at an interpretable and
correctable model of animacy classification. They combine a variety of
classifiers such as the WordNet-based approach of \cite{evans:00},
dictionary sources and systems for named entity recognition.

% commentary on above approaches
The approaches mentioned above present us with a number of
problems. First, almost all of them heavily rely on costly,
linguistically informed features derived from lexical-semantic
databases or syntactic parsing. For most languages in the world,
however, we cannot rely on these resources, either because they do not
exist, or because their performance is insufficient.  Second, animacy
detection is often seen as a useful feature for a range of natural
language processing techniques, such as anaphora resolution and
syntactic parsing. The mutual dependence between these techniques and
animacy detection, however, leads to a chicken and egg
situation. Another problem with the approaches above is that they are
lemma-based, which means that the models are generally insensitive to
different usages of a particular word. A pancake is typically not an
animate entity, but, as we have seen, whether it is animate or not may
depend on the story in which it appears, and even within the story the
animacy may change. We therefore need to develop a dynamic, data
driven system that is token based and does not rely too much on static
and fixed world knowledge.

\section{The Data, Annotation and Preprocessing}\label{sec:data}

To develop this dynamic, data driven system we used a corpus of Dutch
folktales. The corpus consists of 74 Dutch stories from the collection
\textit{Volkssprookjes uit Nederland en Vlaanderen} compiled by
\cite{sinninghe:78}. The collection is made up of Dutch retellings of
popular stories, including such tales as \textit{The Bremen Town
  Musicians} (ATU 130) and \textit{The Table, the Ass, and the Stick }
(ATU 563), as well as lesser-known stories such as \textit{The Singing
  Bone} (ATU 780) and \textit{Cock, Hen, Duck, Pin, and Needle on a
  Journey} (ATU 210). This last story is again a clear example of
otherwise inanimate objects being animate in a folk story, as it
concerns the adventures of several household items, such as a
\textit{pin}, a \textit{hackle}, an \textit{egg}, and a
\textit{whetstone}. A digital version of the collection is available
in the Dutch Folktale Database from the Meertens Institute (corpus
SINVSUNV.20E).\footnote{See \url{http://www.verhalenbank.nl}} Using a
single collection for our corpus presents us with a helpful
homogeneity with regard to the editor, length and language use. On the
other hand, the collection displays diversity as well in the choice of
the stories, which entail fairy tales, legends, and sagas.
%getallen nakijken%
%moet je de labels hier al wel noemen? want dat lijkt me eerder een resultaat dan een pre-processing stap toch?%

The data were labeled by two annotators. We computed an
inter-annotator agreement score (Cohen's Kappa) of $K=0.95$ on the
basis of five stories that were annotated by both annotators. The
following sentence provides an example annotation. All unlabeled words
are implicitly considered to be inanimate.
\begin{exe}
\ex
\gll Jij smid, jij bent de sterkste; hou je vast aan de bovenste
takken, en dan ga jij kleermaker aan zijn benen hangen en zo gaan we maar door\\
    \textsc{anim} \textsc{anim} \textsc{anim} {} {} {} {} \textsc{anim} {} {} {} {} {} {} {} {} \textsc{anim} \textsc{anim}
    {} \textsc{anim} {} {} {} {} {} \textsc{anim} {} {} \\
\trans `You, blacksmith, you are the strongest; hold on to the upper
branches and then, you, tailor, will grab his legs and so we go on\ldots'
\end{exe}
The corpus consists of 74504 words, from 5549 unique words. All word
tokens were assigned a binary label `animate' or `inanimate', where
all words that were not animate (e.g. determiners, adjectives, etc.)
were labelled as inanimate. This resulted in 11542 animate tokens of
743 types, and 62926 inanimate tokens from 5011 unique inanimate
words. A selection of words, such as the abovementioned
\textit{pancake} and \textit{egg}, was labeled as both animate and
inanimate, depending on the context. The distribution of the nouns is
telling: of the animate tokens, 4627 (40,1\%) are nound, while only
6878 of the inanimate tokens (11 \%) are nouns. This shows that being
a noun is already somewhat of an indication for animacy.

After tokenization with the tokenization module of the Python software
package Pattern \cite{smedt:12} we fed all stories to the
state of the art syntactic parser for Dutch, Alpino \cite{bouma:01}.

% gaat nu de volgorde niet fout? Eerst tokenizeren en parsen, en dan animate/inanimate toekennen?%

\section{Experimental Setup}\label{sec:models}
This section describes our experimental setup including the features
used, the machine learning models we applied and our methods of
evaluation.\footnote{The data set and the code to perform the
  experiments are made available online at
  \url{https://fbkarsdorp.github.io/animacy-detection}}.

\subsection{Task description}

We formulate the problem of animacy detection as a classification
problem where the goal is to assign a label at word level, rather than
at lemma level, indicating whether it represents an animate or an
inanimate word.

\subsection{Evaluation}

Animate words are far outnumbered by inanimate words in our collection
(see (\S\ref{sec:data})). Reporting accuracy scores would therefore
provide skewed results, strongly favoring the majority category. The
relative rarity of animate words makes evaluation measures such as the
well-known $F1$-score more appropriate. For all experiments we report
on the precision, recall and $F1$-score \cite{rijsbergen:79}. Also,
while in most of the literature on animacy detection results are only
presented for the classification of nouns or noun phrases, we will
present these and also report on the results for all words in a text.

In real world applications, an animacy detection system will most
likely be faced with completely new texts instead of single words. It
is therefore important to construct a training and test procedure in
such a way that it mimics this situation as closely as possible. If we
would, for example, make a random split of 80\% of the data for
training and 20\% for testing on the word level, we run the risk of
mixing training data with test data, thereby making it too easy for a
system to rely on words it has seen from the same
text. \cite{bowman:12} fall into this trap by making a random split in
their data on the sentence level. It is very likely that in their
setup sentences from the same document are present in both the
training data and the test data, making their evaluation
unrealistic. To circumvent this problem, we split the data at the
story level. We make use of 10-fold cross-validation, where we
randomly select 10\% of the stories for testing and train on the
remaining 90\%.

\subsection{Features}

We explore a range of different features and feature combinations
including lexical features, morphological features, syntactic features
and semantic features.

\subsubsection{Lexical features}
We take a sliding window approach where for each focus word (i.e.\ the
word for which we want to predict whether it is animate or not) we
extract $n$ words to the left, $n$ words to the right and the focus
word itself. In all experiments we set $n$ to 3. In addition to the
word forms, for each word in a window we extract its lemma.

\subsubsection{Morphological Features}
For each word we extract its part-of-speech tag. For reasons of
comparability we choose to use the tags as provided by the output of
the syntactic parser Alpino, instead of a more specialized
part-of-speech tagger. Again, we take a sliding window approach and
extract the part-of-speech tags for three words left and right of the focus word, as well as the tag of the focus word itself.

\subsubsection{Syntactic Features}
We extract the dependency tag for each word and its $n$ neighbors to
the right and to the left as provided by the syntactic parser
Alpino. Animate entities tend to take the position of subject or
object in a sentence which is why this feature is expected and has
proven to perform rather well.

%is n hier weer 3? Je specificeert dat niet%

\subsubsection{Semantic Features} The most innovative feature we have
included in our model is concerned with semantic similarity. In his
\textit{Philophische Untersuchungen} Wittgenstein already suggests
that ``Die Bedeutung eines Wortes ist sein Gebrauch in der Sprache''
(PI 43). The well-known insight in computational linguistics that the
meaning of words can be approximated by comparing the linguistic
contexts in which words appear, is an implementation of this very
idea. The idea is very simple, yet powerful: words that often
co-appear with the same set of words, will have a more similar
meaning. Recently, there has been a lot of interest in procedures that
can automatically induce so-called `word embeddings' from large,
unannotated collections of texts. These models typically attempt to
learn a lower dimensional vector for each word in the vocabulary which
captures the typical co-occurrence patterns of the word in the
corpus. The similarity between words can then be approximated by
applying arithmetics to these vectors of word embeddings.

%je moet een of meerdere bronnen noemen hier, by "recently there has been interest"%

We have trained word embeddings with 300 dimensions using the popular
skip-gram architecture \cite{mikolov:13} on the Dutch corpus of COW
(COrpora from the Web).\footnote{The word embedding vectors have been
  published online at XXX.} COW is a collection of linguistically
processed web corpora for English, Dutch, Spanish, French, Swedish and
German \cite{schaefer:12}. The 2014 Dutch corpus contains 6.8 billion
word tokens. The idea behind using the word embeddings is that
similarities between animate words can be estimated by inspecting the
context in which they occur. From this follows for example that the word
embeddings of an animate word are more similar to those of other
animate words, as opposed to the embeddings of inanimate words. To give an
example, the 5 nearest neighbors of \textit{Assepoester}
(`Cinderella') are: \textit{Sneeuwwitje} (`Snow White') (sim = 0.807),
\textit{Roodkapje} (`Little Red Riding Hood') (sim = 0.747),
\textit{Doornroosje} (`Sleeping Beauty') (sim = 0.715),
\textit{Repelsteeltje} (`Rumpelstiltskin') (sim = 0.669) and
\textit{Alladin} (0.645). In contrast, the 5 most similar words to
\textit{boek} (`book') are: \textit{boekje} (`book-\textsc{dim}')
(sim=0.8), \textit{kinderboek} (`children's book') (0.757),
\textit{stripboek} (`comic') (0.713), \textit{kortverhaal} (`short
story') (0.709), \textit{jeugdboek} (`youth book') (0.706).

We add the word embedding vectors to our animacy model as follows. We
augment each word in the data with its corresponding 300 dimension
word embeddings vector. Out-of-vocabulary words were given a vector in
which all dimensions are set to zero.


\subsection{Models}
We employ a Maximum Entropy classifier with L2 regularization as
implemented in \cite{sklearn}. In all experiments, we set the
regularization strength parameter $C$ to 1.

We compare nine models in which we make use of different feature
combinations: (1) words, (2) words and Part-of-Speech tags, (3)
words, Part-of-Speech tags and lemmata, (4) words, Part-of-Speech
tags, lemmata and dependency tags, (5) word embeddings and (6-9) the
features in model 1 to 4 plus word embeddings.


\section{Results}\label{sec:results}

In Table \ref{tab:results-all} we present the results for all nine
models on the complete data set. For each model we report the
precision, recall, $F1$-score and Area Under the Precision-Recall
Curve (AUC) for the animate words and the
inanimate words.

\begin{table}
\centering
\begin{tabular}{llrrrr}
\toprule
features & class & precision &  recall & $F1$ &  AUC \\
\midrule
\multirow{2}{*}{embeddings}                            & inanimate     &       0.98 &    0.99 &    0.98 & \multirow{2}{*}{0.93} \\
                                                       & animate     &       0.92 &    0.89 &    0.90 &  \\
\multirow{2}{*}{word}                                  & inanimate     &       0.96 &    0.99 &    0.98 & \multirow{2}{*}{0.93} \\
                                                       & animate     &       0.94 &    0.78 &    0.85 &  \\
\multirow{2}{*}{word + embeddings}                     & inanimate     &       0.98 &    0.99 &    0.98 & \multirow{2}{*}{0.96} \\
                                                       & animate     &       0.93 &    0.90 &    0.91 &  \\
\multirow{2}{*}{word + PoS}                            & inanimate     &       0.97 &    0.99 &    0.98 & \multirow{2}{*}{0.95} \\
                                                       & animate     &       0.94 &    0.86 &    0.89 &  \\
\rowcolor{Gray}                                        & inanimate     &       0.98 &    0.99 &    0.99 &  \\
\rowcolor{Gray}\multirow{-2}{*}{word + PoS + embeddings}   & animate     &       0.94 &    0.91 &    0.93 & \multirow{-2}{*}{0.97} \\
\multirow{2}{*}{word + PoS + lemma}                    & inanimate    &       0.97 &    0.99 &    0.98 & \multirow{2}{*}{0.96} \\
                                                       & animate     &       0.94 &    0.86 &    0.90 &  \\
\multirow{2}{*}{word + PoS + lemma + embeddings}       & inanimate     &       0.98 &    0.99 &    0.99 & \multirow{2}{*}{0.97} \\
                                                       & animate     &       0.94 &    0.91 &    0.93 &  \\
\multirow{2}{*}{word + PoS + lemma + dep}              & inanimate     &       0.97 &    0.99 &    0.98 & \multirow{2}{*}{0.96} \\
                                                       & animate     &       0.94 &    0.86 &    0.90 &  \\
\multirow{2}{*}{word + PoS + lemma + dep + embeddings} & inanimate     &       0.98 &    0.99 &    0.99 & \multirow{2}{*}{0.97} \\
                                                       & animate     &       0.94 &    0.92 &    0.93 &  \\
\bottomrule
\end{tabular}
\caption{Precision, recall, $F1$ and Area Under the Curve (AUC) scores for animate and inanimate classes per feature setting for all words.}
\label{tab:results-all}
\end{table}


All models perform well on classifying inanimate words. However, since
this is the majority class, it is of course far more interesting to
compare the performance of the models on the animate instances. It is
interesting to observe that the `simple' $n$-gram word model already
performs rather well. Adding more features, such as Part-of-Speech,
lemmata, etcetera, only has a positive impact on the recall of the
model, while leaving the precision untouched. As can be observed from
the table, employing the rather expensive dependency features shows
barely any improvement.

The model that only uses word embedding features is one of the best
performing models. This is a context insensitive model that operates
on the level of the vocabulary, which means that it will predict the
same outcome for each token of a particular word type. The high
precision and high recall show us that this model has acquired
knowledge about which words \textit{typically} group with animate
words and which with inanimate words. However, the models that combines
the word embeddings with the context sensitive features, such as word
$n$-grams or Part-of-Speech tags, outperform the context insensitive
model.

The best performance is
achieved by the model that combines the word features, part-of-speech
tags and the word embeddings with an $F1$-score of 0.93 for animate
words and 0.99 on inanimate words. Adding more features does not
result in any more performance gain.

Table \ref{tab:results-noun} presents the results for all nouns and
names in the data set. The best performance is again achieved by the
model that combines the word features with the Part-of-Speech tags and
word embeddings, resulting in an $F1$-score of 0.92 for animate
instances and 0.94 for inanimate instances.


\begin{table}
\centering
\begin{tabular}{llrrrr}
\toprule
features & class &  precision &  recall & $F1$ &  AUC \\
\midrule
\multirow{2}{*}{embeddings}                            & inanimate     &       0.89 &    0.94 &    0.92 & \multirow{2}{*}{0.92} \\
                                                       & animate       &       0.92 &    0.84 &    0.88 &  \\
\multirow{2}{*}{word}                                  & inanimate     &       0.78 &    0.98 &    0.87 & \multirow{2}{*}{0.91} \\
                                                       & animate       &       0.96 &    0.60 &    0.74 &  \\
\multirow{2}{*}{word + embeddings}                     & inanimate     &       0.90 &    0.96 &    0.93 & \multirow{2}{*}{0.96} \\
                                                       & animate       &       0.94 &    0.86 &    0.89 &  \\
\multirow{2}{*}{word + pos}                            & inanimate     &       0.86 &    0.96 &    0.90 & \multirow{2}{*}{0.94} \\
                                                       & animate       &       0.93 &    0.78 &    0.84 &  \\
\rowcolor{Gray}                                        & inanimate     &       0.93 &    0.96 &    0.94 &  \\
\rowcolor{Gray}\multirow{-2}{*}{word + PoS + embeddings} & animate       &       0.94 &    0.89 &    0.92 &\multirow{-2}{*}{0.97}  \\
\multirow{2}{*}{word + PoS + lemma}                    & inanimate     &       0.87 &    0.96 &    0.91 & \multirow{2}{*}{0.95} \\
                                                       & animate       &       0.94 &    0.80 &    0.86 &  \\
\multirow{2}{*}{word + PoS + lemma + embeddings}       & inanimate     &       0.92 &    0.96 &    0.94 & \multirow{2}{*}{0.97} \\
                                                       & animate       &       0.94 &    0.89 &    0.92 &  \\
\multirow{2}{*}{word + PoS + lemma + dep}              & inanimate     &       0.87 &    0.96 &    0.91 & \multirow{2}{*}{0.95} \\
                                                       & animate       &       0.93 &    0.80 &    0.86 &  \\
\multirow{2}{*}{word + PoS + lemma + dep + embeddings} & inanimate     &       0.93 &    0.96 &    0.94 & \multirow{2}{*}{0.97} \\
                                                       & animate       &       0.94 &    0.89 &    0.92 &  \\
\bottomrule
\end{tabular}
\caption{Precision, recall, $F1$ and Area Under the Curve (AUC) scores for animate and inanimate classes
  per feature settings for all words tagged as noun.}
\label{tab:results-noun}
\end{table}


\section{Concluding Remarks}

The approach taken in this paper to create a model for animacy
classification using linguistically uninformed features proves to be
successful. We compared the performance of linguistically informed
models (using features such as part-of-speech and dependency tags) to
models that make use of lower dimensional representations of the
data. With the exception of the model that solely makes use of these
representations, all models benefit from adding these features. The
model that requires the least linguistic information (word $n$-grams
plus word embeddings) outperforms all linguistically informed
models. The best results are reported by the model that combines word
$n$-grams with Part-of-Speech $n$-grams and word embeddings.

We have the following recommendation for future research. Natural
language processing models such as co-reference resolution or
linguistic parsing could benefit from a module that filters animate
from inanimate candidate words. Since these models typically depend
heavily on linguistic features, it is important that additional
features, such as animacy, are not dependent on these features as
well. Our linguistically uninformed model for animacy detection
provides such a module.



\subparagraph*{Acknowledgments}

The work on which this paper is based has been supported by the
Computational Humanities Programme of the Royal Netherlands Academy of
Arts and Sciences, under the auspices of the Tunes \& Tales
project. For further information, see \url{http://ehumanities.nl}.

\bibliography{paper}

\end{document}
%%% Local Variables:
%%% mode: latex
%%% TeX-master: t
%%% End:
